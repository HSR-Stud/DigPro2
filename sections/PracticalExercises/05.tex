\subsection{Exercise 5}
Write a Matlab script that implements the Marr-Hildreth algorithm. 
Use the result from problem 10.19 to speed up your implementation and compare it to an implementation which  uses 2D convolution.
How much faster is the 1D approach? Does it result in exactly the same values?

\begin{lstlisting}
% Marr Hilderth Algorithm

%% Settings:
sigma = 4;
laplacian = [1 1 1; 1 -8 1; 1 1 1];

%% Bild laden
I = double(imread('cameraman.tif'));
[Ix, Iy] = size(I);
maskSize = floor(ceil((6 * sigma))/2)*2 + 1;

%% Maske generieren (LoG)
mask = zeros(maskSize,maskSize);
for a = 1:maskSize
    for b = 1:maskSize
        x = a - double(int8(maskSize / 2));
        y = b - double(int8(maskSize / 2));
        mask(a,b) = exp(-(x^2+y^2)/(2*sigma^2));
    end
end
mask = conv2(mask,laplacian, 'same');

%% Bild mit Maske falten und Zero-Crossing berechnen
% Zeitmessung Start
tic
I2 = conv2(I, mask, 'same');
% Zero crossing
I3 = zeros(Ix,Iy);
for a = 2:Ix-1
    for b = 2:Iy-1
        value = (sign(I2(a-1,b-1)) * sign(I2(a+1, b+1))) + (sign(I2(a-1,b)) * sign(I2(a+1, b))) + (sign(I2(a,b-1)) * sign(I2(a, b+1))) + (sign(I2(a+1,b-1)) * sign(I2(a-1, b+1)));
        if(value < 0)
            I3(a,b) = 1;
        end;        
    end
end
% Zeitmessung Ende
toc
imshow(I3);
\end{lstlisting}
