\subsubsection{Exercise 10}
Write a Matlab script that calculates for a given 8 bit gray scale image the values in the table 11.2. Create an artificial test set of images, one random, one smooth and one periodic and calculate the entries in the table for each image. Could you use these regional descriptors to distinguish the different images?
\\
\textbf{Solution}
\begin{lstlisting}
% Drei Bilder generieren. Der einfachheit halber alle gleich gross
Irandom = rand([100 100]);
Ismooth = repmat(1:100,100,1)/100;
Iperiodic = repmat([zeros(100,10) ones(100,10)],1,5);

% Show images
subplot(2,3,1);
imshow(Irandom);
title('Random Image');
subplot(2,3,4);
imhist(Irandom);
title('Hist Random Image');
subplot(2,3,2);
imshow(Ismooth);
title('Smooth Image');
subplot(2,3,5);
imhist(Ismooth);
title('Hist Smooth Image');
subplot(2,3,3);
imshow(Iperiodic);
title('Periodic Image');
subplot(2,3,6);
imhist(Iperiodic);
title('Hist Periodic Image');

% Werte in Tabelle 11.2 berechnen
% Mean
MeanRand = mean(Irandom(:))
MeanSmooth = mean(Ismooth(:))
MeanPeriodic = mean(Iperiodic(:))
% Standard Deviation
StDevRand = std(Irandom(:))
StDevSmooth = std(Ismooth(:))
StDevPeriodic = std(Iperiodic(:))
% Variance
VarRand = var(Irandom(:))
VarSmooth = var(Ismooth(:))
VarPeriodic = var(Iperiodic(:))
% R normalized
RRand = 1-1/(1+VarRand)
RSmooth = 1-1/(1+VarSmooth)
RPeriodic = 1-1/(1+VarPeriodic)
% Third moment
[CountRand XRand] = imhist(Irandom);
TMRand = 0;
UniRand = 0;
EntropyRand = 0;
for i = 1:256
    TMRand = TMRand + ((XRand(i)-MeanRand).^3 * CountRand(i)/100^2);
    UniRand = UniRand + (CountRand(i)/100^2)^2;
    EntropyRand = EntropyRand + (CountRand(i)/100^2) .* log2(CountRand(i)/100^2);
end
TMRand

[CountSmooth XSmooth] = imhist(Ismooth);
TMSmooth = 0;
UniSmooth = 0;
EntropySmooth = 0;
for i = 1:256
    TMSmooth = TMSmooth + ((XSmooth(i)-MeanSmooth).^3 * CountSmooth(i)/100^2);
    UniSmooth = UniSmooth + (CountSmooth(i)/100^2)^2;
    EntropySmooth = EntropySmooth + (CountSmooth(i)/100^2) .* log2(CountSmooth(i)/100^2);
end
TMSmooth

[CountPeriodic XPeriodic] = imhist(Iperiodic);
TMPeriodic = 0;
UniPeriodic = 0;
EntropyPeriodic = 0;
for i = 1:256
    TMPeriodic = TMPeriodic + ((XPeriodic(i)-MeanPeriodic).^3 * CountPeriodic(i)/100^2);
    UniPeriodic = UniPeriodic + (CountPeriodic(i)/100^2)^2;
    EntropyPeriodic = EntropyPeriodic + (CountPeriodic(i)/100^2) .* log2(CountPeriodic(i)/100^2);
end
TMPeriodic
% Unifomity
UniRand
UniSmooth
UniPeriodic
% Entropy
EntropyRand = -EntropyRand
%EntropySmooth = -EntropySmooth
%EntropyPeriodic = -EntropyPeriodic
\end{lstlisting}