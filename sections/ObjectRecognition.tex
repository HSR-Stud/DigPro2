\section{Object Recognition\buch{Ch. 12}}
\begin{itemize}
\item Recognition of \emph{individual} image regions called \emph{object} or \emph{pattern}
\item Concept of ``learning'' from sample patterns
\end{itemize}

\subsection{Patterns and Pattern Classes\buch{Ch. 12.1}}
\begin{itemize}
	\item A \emph{pattern} is an arrangement of descriptors
		(as in \ref{sec:representationDescription}) in pattern classification
		called \emph{features}.
	\item A \emph{pattern class} is a set of patterns, denoted $\omega_1, \omega_2, ..., \omega_W$ W is the number of classes
	\item The goal is to assign a given patter into a class
\end{itemize}

\subsubsection{Vectors}
\begin{itemize}
	\item Used for quantitative descriptors
	\item every dimension contains the numerical value of a descriptor
	\item $x =
		\begin{bmatrix}
			x_1 \\
			x_2 \\
			\vdots \\
			x_n \\
		\end{bmatrix}$
\end{itemize}
Example: Classifying flowers based on petal length and width into one of three
classes.

\subsubsection{Strings}
Structural information can be captured in strings. Example
\nameref{sec:boundaryDescriptors}
\begin{itemize}
	\item Connectivity patterns (order is important)
	\item Compact description, better than sampling into a feature vector
\end{itemize}

\subsubsection{Tree}
If there is hierarchy then a descriptor which is based on a tree structure
should be used.

% TODO: Tikz Tree einfügen

\subsection{Recognition Based on Decision Theory\buch{Ch. 12.2}}
\paragraph{Discriminant functions}
The core of decision theory $d_i(x)$ defined for each class $\omega_i$.
These are not known and finding them is the main goal.

A class $\omega_i$ is selected if
\begin{align*}
d_i(x) > d_j(x) && j=1,2, \cdots, W; j \neq i
\end{align*}
The interesting parts are decision boundaries between classes
\begin{align*}
	d_i(x) - d_j(x) = 0
\end{align*}

\subsubsection{Matching\buchSeite{866}}
Each class is represented by prototypes.
A vector $x$ belongs to the class with the shortest distance.
% TODO: Tikz Diagramm mit klassen, prototypen und grenzen
\paragraph{Minimum distance classifier}
The mean vector of a class is a powerful prototype vector.
\begin{align*}
	m_j = \frac{1}{N_j} \sum_{x\in \omega_j}x_j && j = 1,2,\dots, W
\end{align*}
A simple matching function is the Euclidean distance.
\begin{align*}
	D_i(x) = ||x-m_j|| = \left(x^Tm_j\right)^{1/2}
\end{align*}
The resulting discriminant function which is the largest, when the distance is
the smallest.
\begin{align*}
	d_j(x) = x^Tm_j - \frac{1}{2}m_j^Tm_j
\end{align*}
The decision boundary between class $\omega_i$ and $\omega_j$ will be
\begin{align*}
	d_{ij}(x) &= d_i(x) - d_j(x)\\
	&= x^t(m_i-m_j) - \frac{1}{2}(m_i-m_j)^T(m_i+m_j) = 0
\end{align*}
Which is a line through the midpoint of $m_i$ and $m_j$ for $n=2$.
For $n=3$ its a plane and for $n>3$ it's a hyperplane.
% TODO: Tikz mit diagram
% TODO: Evt. Beispiel?

\paragraph{Matching by correlation\buchSeite{869}}
Calculate cross-correlation between a mask $w(x,y)$ of size $m$ $n$ and an image
$f(x,y)$ to find the mask in the image.
\begin{align*}
	c(c,y) = \sum_s\sum_tw(s,t)f(x+s,y+t)
\end{align*}
This is sensitive to contrast and average intensity in $f$ and $w$.
Hence the normalized cross correlation is used
\begin{align*}
	\gamma(x,y) = \frac{\sum_s\sum_t\left[w(s,t)-\bar{w}\right]\left[f(x+s,y+t)-\bar f_{xy}\right]}
	{
	\left\lbrace 
		\sum_s\sum_t\left[w(s,t)-\bar w\right]^2
		\sum_s\sum_t\left[f(x+s,y+t)-\bar f_{xy}\right]^2
	\right\rbrace^{\frac{1}{2}}
	}
\end{align*}
$y(x,y)$ is always in range $[-1, 1]$, where +1 is perfect match and -1 a perfect anti-match.
This is sensitive to scale and rotation.


\subsection{Structural Methods\buch{Ch. 12.3}}

