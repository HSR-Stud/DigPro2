\section{Object Recognition\buch{Ch. 12}}
\begin{itemize}
\item Recognition of \emph{individual} image regions called \emph{object} or \emph{pattern}
\item Concept of ``learning'' from sample patterns
\end{itemize}

\subsection{Patterns and Pattern Classes\buch{Ch. 12.1}}
\begin{itemize}
	\item A \emph{pattern} is an arrangement of descriptors
		(as in \ref{sec:representationDescription}) in pattern classification
		called \emph{features}.
	\item A \emph{pattern class} is a set of patterns, denoted $\omega_1, \omega_2, ..., \omega_W$ W is the number of classes
	\item The goal is to assign a given patter into a class
\end{itemize}

\subsubsection{Vectors}
\begin{itemize}
	\item Used for quantitative descriptors
	\item every dimension contains the numerical value of a descriptor
	\item $x =
		\begin{bmatrix}
			x_1 \\
			x_2 \\
			\vdots \\
			x_n \\
		\end{bmatrix}$
\end{itemize}
Example: Classifying flowers based on petal length and width into one of three
classes.

\subsubsection{Strings}
Structural information can be captured in strings. Example
\nameref{sec:boundaryDescriptors}
\begin{itemize}
	\item Connectivity patterns (order is important)
	\item Compact description, better than sampling into a feature vector
\end{itemize}

\subsubsection{Tree}
If there is hierarchy then a descriptor which is based on a tree structure
should be used.

% TODO: Tikz Tree einfügen

\subsection{Recognition Based on Decision Theory\buch{Ch. 12.2}}

\subsection{Structural Methods\buch{Ch. 12.3}}

