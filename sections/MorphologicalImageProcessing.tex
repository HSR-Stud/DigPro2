\section{Morphological Image Processing\buch{Ch. 9}}
\subsection{Set Theory}
\subsubsection{Reflection}
\[
	\hat{B} = \{w | w = -b,\quad \text{for} \quad b \in B \}
\]


\subsubsection{Translation}
\[
	(B)_z = \{ c | c = b + z,\quad \text{for} \quad b \in B\}
\]


\subsubsection{Erosion}
With A and B as sets in $Z^2$, the erosion of A by B is defined as
\[
	A \ominus B = \{z|(B)_z \subseteq A  \}
\]

This is the set of all translations by z such that the set B is still a subset of A.

Equivalently, if B must be contained in A, it is not allowed to intersect with the background:
\[
	A \ominus B = \{z|(B)_z \cap A^c = \varnothing  \}
\]


\subsubsection{Dilation}
With A and B as sets in $Z^2$, the dilation of A by B is defined as
\[
	A \oplus B = \{z |(\hat{B})_z \cap A \neq \varnothing \}
\]
This is the set of all translations by z such that the set B reflected about its origin has at leas one overlap with the set A.

Equivalently, since the intersection is not allowed to be empty and all elements in it must be from A:
\[
	A \oplus B = \{z | [(\hat{B})_z \cap A] \subseteq A \}
\]


\subsubsection{Duality erosion/dilation}
Erosion and dilation are dual:
\[
	(A  \ominus B)^c = A^c \oplus \hat{B}
\]
\[
	(A \oplus B)^c = A^c \ominus \hat{B}
\]


\subsubsection{Opening and closing}
The opening of A by B is defined: (first erosion then dilation)
\[
	A \circ B = (A \ominus B) \oplus B
\]

The closing of A by B is defined: (first dilation then erosion)
\[
	A  \bullet B = (A \oplus B) \ominus B
\]

Opening and closing are dual operations: 
\[
	(A  \bullet B)^c = (A^c \circ \hat{B})
\]
\[
	(A \circ B)^c = (A^c \bullet \hat{B})
\]


\subsubsection{The hit-or-miss transform}
The hit-or-miss transform is a basic tool for shape detection. Basic idea:
\begin{itemize}
	\item Through erosion of the image by the object, possible locations for the object in the image are left, since elements smaller than the objects disappear
	
	\item Through erosion of the background of the image with the local background of the object, possible locations for the object in the image are left
	
	\item The intersection of these possible locations result in a reliable  detection of the object location
	\[
		A \circledast B = (A \ominus D) \cap [ A^c \ominus (W-D)]
	\]
	
	\item In general, if B1 is the object and B2 is the corresponding background this can be written
	\[
		A \circledast B = (A \ominus B_1)\cap (A^c \ominus B_2)	
	\]	
\end{itemize}



